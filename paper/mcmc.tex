\chapter{MCMC}

This chapter covers Bayesian updating via computer. The idea is simple: begin with a prior
distribution that describes your beliefs about the data, and then update with new
information from another distribution of the data to arrive at a posterior distribution.

The technical details of the process regard describing distributions
in a computer-friendly manner, and making random draws from those distributions.

\section{Random number generation}

\paragraph{The RNG module}

\paragraph{Drawing from a textbook distribution}
\label{randomnumbers}
Once you have an RNG initialized, then drawing from a common distribution is just a matter
of passing the RNG structure to the appropriate function. Here are the two you will be most
likely to use; a dozen others are listed in the GSL manual.
\begin{verbatim}
#include <gsl/gsl_randist.h>
double gsl_ran_gaussian (const gsl_rng * R, double SIGMA);
gsl_ran_bivariate_gaussian (const gsl_rng * R, double
          SIGMA_X, double SIGMA_Y, double RHO, double * X, double * Y)
\end{verbatim}
Notice that the univariate distribution returns a value directly, but
the bivariate version requires that you send in pointers to the variables
({\tt X} and {\tt Y}) that will be given the random draws.

\paragraph{Drawing from your own distributions}


\paragraph{Seeds}

\section{Describing distributions}

\paragraph{The histogram module}

\section{Sampling and updating}



\paragraph{Gibbs Sampling}

